\clearpage

\setcounter{chapter}{3}
\setcounter{section}{0}
% Add the chapter to table of contents
\addcontentsline{toc}{chapter}{\numberline{3}From Theory to Practice - Experimental Pathways}


% Set up page style for this chapter (assuming fancyhdr is loaded in preamble)
\pagestyle{fancy}
\fancyhf{} % Clear all header and footer fields
\fancyhead[L]{\footnotesize\textit{Ars Post Faber: Digital Fabrication Democratization Through Embodied Knowledge Preservation}}
\fancyfoot[C]{\thepage} % Page number in footer center
\renewcommand{\headrulewidth}{0pt}
\renewcommand{\footrulewidth}{0pt}


% Custom chapter title to match abstract formatting
\noindent
{\Large\textbf{Chapter 3: Developing Theory From Practice, Experimental Pathways}}
\vspace{0.3cm}
\hrule
\vspace{0.8cm}
\label{ch:experimental_pathways}

% Set no paragraph indentation
\setlength{\parindent}{0pt}
The theoretical framework developed in the previous chapters suggested that genuine democratization of digital fabrication requires preservation-based approaches that maintain adaptive authority rather than simply expanding access to predetermined tools. Yet this analysis raised questions about implementation: \textit{How might such preservation actually operate within existing technological contexts?} \textit{Could alternative "interfaces" bridge the gap between computational precision and embodied expression?} The following interventions sought to address these questions through practical experiments that tested the theoretical propositions in real-world making contexts.

\vspace{0.5cm}

This chapter will document different "interventions" that progressively refined approaches to preserving the embodied knowledge within digital workflows. Each experiment built upon insights from the previous, leading to the development of Ars Post Faber, the open-source Grasshopper\footnote{Grasshopper is a visual programming language and environment that runs within the Rhinoceros 3D computer CAD software. Developed by David Rutten at Robert McNeel \& Associates, Grasshopper enables users to build generative algorithms through a node-based interface without requiring traditional programming knowledge, making it widely used in parametric design, digital fabrication, and computational design workflows.} plugin\footnote{A plugin (also called an add-on or extension) is a software component that adds specific functionality to an existing program. In the context of CAD and design software, plugins extend the applications capabilities by providing new tools, commands, or workflows. They are typically developed by third parties and can be installed and removed without modifying the main software, allowing users to customize their design environment for specific tasks or methodologies.} that embodies the research's theoretical conclusions.

\section{\textit{CR3ATED}: Reimagining CAD Interfaces for Artisan Expression}
\textit{"If craftsmanship's essence lies in the creative problem solving process, how can digital fabrication tools become active participants in it rather than automation devices?"}

\vspace{0.5cm}

\subsection{Exploring Alternative Interface Approaches}

Building upon insights from AI.RTISANSHIP, it was concluded that computational systems could successfully capture and analyze certain patterns of skilled movement, yet this technical capability highlighted a bigger challenge where the preserved data represented only "surface" manifestations of embodied knowledge rather than the adaptive reasoning processes. This raised a different question: \textit{Rather than attempting to extract embodied knowledge from practitioners, what if digital tools could be designed to better support and amplify the adaptive decision-making processes that characterize skilled practice?}

\vspace{0.5cm}

This inquiry led to the development of \textit{CR3ATED}, a web-based application designed to test alternative interface approaches within digital design and fabrication workflows. Where AI.RTISANSHIP looked to decode existing craft knowledge through computational analysis, \textit{CR3ATED} explored how interface design itself might preserve creative expression by exploring different human-software interactions.

\vspace{0.5cm}

The \textit{CR3ATED} experiment investigated whether alternative modes of human-software interaction could maintain the continuity between creative intention and material execution that conventional CAD workflows tend to fragment. The central hypothesis was that interface design actively constructs the kinds of creative relationships that become possible within digital fabrication workflows, suggesting that preserving craft agency might require different approaches to how makers engage with computational design tools.

\subsection{Craftinnova: Testing Alternative Interfaces}

The CR3ATED web application was developed specifically for the "Nuevos Métodos de Aprendizaje Artesano" workshop at CRAFTINNOVA\footnote{CRAFTINNOVA is Spain's national event combining artistic crafts and innovation, held annually in Valladolid. The event brings together creators, designers, manufacturers, technology experts, digital fabrication professionals, artists, entrepreneurs, and makers to explore the intersection of traditional craftsmanship with digital technologies.}, creating an opportunity to test these alternative interface approaches with practicing artisans in a real-world context. The workshop brought together traditional craftspeople with clay 3D printing technology, creating an ideal laboratory for investigating how interface design affects the preservation of creative expression.

\vspace{0.5cm}

Rather than using conventional CAD software, the webapp, implemented a touch-based sketching interface that allowed participants to create 3D models through direct manipulation on mobile devices. The application translated 2D sketches into revolution surfaces suitable for clay printing, maintaining the intuitive hand-to-head connection that characterizes crafts while engaging with the digital fabrication capabilities of the machine.

\vspace{0.5cm}

This approach acknowledged that conventional CAD systems fail to express the imperfection and responsiveness from the craft practice. By creating a workflow that began with tactile sketching and culminated with material engagement, the experiment explored how technology might contribute to craft preservation by becoming integrated into evolving creative practice rather than replacing traditional methods.

\subsection{Material Continuity and Digital Mediation}

An important aspect of the workflow developed involved preserving material continuity between digital design and physical fabrication. The clay printer, despite its digital control system, interacts with clay according to the same physical principles that govern hand techniques, unlike the normally used polymers in additive manufacturing that cannot be hand-treated. Clay maintains its material properties regardless of whether it is shaped by hand or extruded through a mechanical nozzle, maintaining material coherence across the digital/physical boundary.

\vspace{0.5cm}

This materiality creates productive constraints rather than arbitrary limitations. Workshop participants encountered the clay printer's capabilities and limitations as a new set of material conditions to navigate, drawing on their traditional knowledge while developing new skills specific to the printer. Unlike conventional CAD/CAM workflows that abstract away material properties through geometric representation, the \textit{CR3ATED} approach maintained material feedback throughout the process.

\vspace{0.5cm}

The touch interface proved to be important for preserving this material relationship. Whereas CAD requires learning abstract geometric manipulation techniques, sketching with the finger or a stylus maintains the tactile connection between hand movement and form development. Participants of the workshop were able to leverage their existing skills while engaging with digital fabrication, rather than having to master entirely new representational systems.

\subsection{Expanding vs. Replacing Practice}

The workshop generated diverse responses from the participants, showing different ways that digital tools might relate to traditional practice. For most, the clay printing process represented a new exploration rather than a shift in their approach. These artisans engaged with the technology as they might with any new tool or technique, integrating it within their existing practice. However, other participants experienced the workshop as an expansion of their literacy in both digital and material domains. For these makers, the interface exploration enabled forms of creative expression and production that would have been difficult (or impossible) to achieve through either traditional hand-building or conventional CAD approaches.

\vspace{0.5cm}

At it's core, the workshop attempted to challenge the binary thinking that has historically characterized discussions of technology and craft. Rather than facing off hand-building versus digital printing as competing methodologies, the comparison between techniques became "an exploration of their complementary strengths" rather than a contest between "superior" and "inferior" approaches. Aligning with Gershenfeld's vision of personal fabrication enabling the convergence of industrial production with personal expression, "which would merge with digital design, to bring common sense and sensibility to the creation and application of advanced technologies" (Gershenfeld, 2007). The clay printer became not merely a tool but a participant in creative dialogue, presenting its own challenges and possibilities while requiring new skills that built upon traditional knowledge.

\vspace{0.5cm}

\subsection{Success and Limitations of the Experiment}

Despite the "success" in exploring a new way of interaction and preserving material relationships, the intervention highlighted certain limitations that pointed towards the need for more sophisticated future experiments. While the touch-based sketching approach built upon CAD's geometric interaction constraints, it remained limited to revolution forms suitable for lathe operations. Participants could only create objects that could be generated by rotating a 2D profile around an axis.

\vspace{0.5cm}

More importantly, the simplification required to make this interface accessible to users with no CAD experience eliminated some of the adaptive capabilities natural to both craft and digital design. The application could capture the gestural sketching attempts, but could not "accommodate" the modifications and iterative refinements that enabled to respond to emerging contexts.

\vspace{0.5cm}

This constraint highlighted the biggest inconvenience of this research: tools simple enough for broad accessibility may lack the expressive range needed to preserve adaptive authority, while tools sophisticated enough to support complex creative decision-making may require technical expertise that creates new barriers to access.

\subsection{Implications of the Simplification Constraints}

CR3ATED experiment most important insight emerged from recognizing that its limitations were not inherent to alternative interface design, but specific to the simplified application context. While the experiment showcased more intuitive approaches to interact with digital design and fabrication, it also highlighted constraints imposed by simplification strategies commonly employed in educational contexts.

\vspace{0.5cm}

This finding, contextualized within broader research on digital fabrication in educational environments, reveal that "focus on the potentials of these technologies has mainly been on the support to STEM oriented learning goals" \citep{smith2016}. At the same time, research specifically examining simplified tools like TinkerCAD shows that while such platforms enable users to "easily build a virtual model and make it tangible" \citep{barbosa2024}, participants recognize "challenges in the use of these technological resources" around creative expression and adaptive modification \citep{barbosa2024}, leading to the realization that rather than creating new simplified tools that inevitably constrain expression, preservation-based democratization might be better achieved by developing more human-centered ways of interacting with sophisticated tools that people already know and use. Professional CAD environments possess the computational sophistication needed to support complex creative decision-making, but their interfaces often fail to support the fluid, responsive workflows that characterize unified agency.

\vspace{0.5cm}

This insight was reinforced by the workshop participants. Those who engaged most with the digital design and fabrication processes in their day to day work were able to foresee how to integrate it within their existing frameworks, suggesting that effective preservation requires tools that augment craft knowledge, maintaining the conditions for adaptive response while leveraging contemporary technological capabilities.

\subsection{Moving Towards Integrated Solutions}







\vspace{0.5cm}
\vspace{0.5cm}
\vspace{0.5cm}
\vspace{0.5cm}
\vspace{0.5cm}
\vspace{0.5cm}
\vspace{0.5cm}
\vspace{0.5cm}
\vspace{0.5cm}
\vspace{0.5cm}
\vspace{0.5cm}
\vspace{0.5cm}
\vspace{0.5cm}
\vspace{0.5cm}


