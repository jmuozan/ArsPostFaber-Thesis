\clearpage

\setcounter{chapter}{3}
\setcounter{section}{0}
% Add the chapter to table of contents
\addcontentsline{toc}{chapter}{\numberline{3}From Theory to Practice - Experimental Pathways}


% Set up page style for this chapter (assuming fancyhdr is loaded in preamble)
\pagestyle{fancy}
\fancyhf{} % Clear all header and footer fields
\fancyhead[L]{\footnotesize\textit{Ars Post Faber: Digital Fabrication Democratization Through Embodied Knowledge Preservation}}
\fancyfoot[C]{\thepage} % Page number in footer center
\renewcommand{\headrulewidth}{0pt}
\renewcommand{\footrulewidth}{0pt}


% Custom chapter title to match abstract formatting
\noindent
{\Large\textbf{Chapter 3: From Theory to Practice - Experimental Pathways}}
\vspace{0.3cm}
\hrule
\vspace{0.8cm}
\label{ch:experimental_pathways}

% Set no paragraph indentation
\setlength{\parindent}{0pt}
The theoretical framework developed in the previous chapters suggested that genuine democratization of digital fabrication requires preservation-based approaches that maintain adaptive authority rather than simply expanding access to predetermined tools. Yet this analysis raised questions about implementation: \textit{How might such preservation actually operate within existing technological contexts?} \textit{Could alternative "interfaces" bridge the gap between computational precision and embodied expression?} The following interventions sought to address these questions through practical experiments that tested the theoretical propositions in real-world making contexts.

\vspace{0.5cm}

This chapter will document different "interventions" that progressively refined approaches to preserving the embodied knowledge within digital workflows. Each experiment built upon insights from the previous, leading to the development of Ars Post Faber, the open-source Grasshopper\footnote{Grasshopper is a visual programming language and environment that runs within the Rhinoceros 3D computer CAD software. Developed by David Rutten at Robert McNeel \& Associates, Grasshopper enables users to build generative algorithms through a node-based interface without requiring traditional programming knowledge, making it widely used in parametric design, digital fabrication, and computational design workflows.} plugin\footnote{A plugin (also called an add-on or extension) is a software component that adds specific functionality to an existing program. In the context of CAD and design software, plugins extend the applications capabilities by providing new tools, commands, or workflows. They are typically developed by third parties and can be installed and removed without modifying the main software, allowing users to customize their design environment for specific tasks or methodologies.} that embodies the research's theoretical conclusions.

\section{CR3ATED: Reimagining CAD Interfaces for Artisan Expression}
\textit{"If craftsmanship's essence lies in the creative problem solving process, how can digital fabrication tools become active participants in it rather than automation devices?"}

\vspace{0.5cm}














