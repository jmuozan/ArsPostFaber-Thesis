% Header with advisor names
\noindent
\begin{minipage}[t]{0.5\textwidth}
\raggedright
Theiss Advisor: Jessica Carmen Guy 
\end{minipage}%
\begin{minipage}[t]{0.5\textwidth}
\raggedleft
Jorge Muñoz Zanón
\end{minipage}
\vspace{1cm}

% Abstract title - left aligned with padding and underline
\noindent
{\Large\textbf{Abstract}}
\vspace{0.3cm}
\hrule
\vspace{0.8cm}

% Set no paragraph indentation for the abstract content
\setlength{\parindent}{0pt}

Digital fabrication technologies have democratized access to production tools while perpetuating the industrial era separation between design conception and material execution. This division, which has historically diminished artisanship by fragmenting holistic creative processes, continues to manifest itself in contemporary CAD/CAM workflows that benefit computational precision over embodied knowledge and tacit decision-making central to craftship, often reducing creation to pure geometry, failing to preserve the material relationships and adaptive responses that characterize traditional making practices.

\vspace{0.5cm}

This research challenges the assumption that fabrication democratization is achieved solely through access to scaled-down industrial tools and instead, looks out to do a reimagination of the relationship between maker, material, and technology, seeking to restore the holistic nature of creative practice within digital contexts, addressing how to preserve creative agency, embodied knowledge, and capacity for personal expression in digital fabrication contexts.

\vspace{0.5cm}

Through experimental tool testing and digital fabrication workshops with artisans and makers, this research develops \textit{Ars Post Faber}, an open-source Grasshopper plug-in within the Rhinoceros CAD environment that approaches thinking and making as an integrated practice. The plug-in implements utilities designed to facilitate fluid Human-Software-Machine interactions, looking to enable embodied expression, contextual adaptation, and tacit knowledge to flow throughout the making process. Rather than abstracting away the creative journey, this approach looks to preserve the complete narrative of creation, including modifications, errors, and decision points, as integral components of the final work.

\vspace{0.5cm}

By attempting to bridge the gap between digital design and material execution, this research looks to contribute to evolving discussions around craft, technology, and creative agency in the digital age, suggesting that true democratization might require representational frameworks that honor the complexity and continuity of human creative processes.

\vspace{1cm}

\noindent
\textbf{Keywords:} craftship, digital fabrication, human-machine interactions, preservation, democratization, artisanship

\vspace{\fill}