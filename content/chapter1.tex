% Set chapter counter to 1 and reset section counter
\setcounter{chapter}{1}
\setcounter{section}{0}

% Custom chapter title to match abstract formatting
\noindent
{\Large\textbf{Chapter 1: Introduction}}
\vspace{0.3cm}
\hrule
\vspace{0.8cm}
\label{ch:introduction}

% Set no paragraph indentation
\setlength{\parindent}{0pt}

The transformation of making practices from medieval workshops to contemporary digital fabrication represents a reconfiguration of how creative decisions flow through productive processes. To understand current challenges in digital fabrication, it becomes essential to trace how creative agency, defined as 'meaningful intentional action' (Niedderer \& Townsend 2024) that enables makers to exercise decision-making authority throughout the making process, has been systematically redistributed across different historical moments and technological contexts.

\vspace{0.5cm}

Medieval artisan guilds operated through integrated knowledge systems where individual craftspeople maintained comprehensive understanding of their entire productive domain. As Richardson (2008) notes, these craft guilds were 'organized along trade lines' with members who 'shared religious observances and fraternal dinners,' creating communities where 'guilds ensured production standards were maintained' through collective oversight of the complete production process. The master carpenter knew not only how to shape wood but why specific joints were chosen, when to adapt techniques for different grain patterns, and how environmental conditions would affect long-term structural integrity. This integration of conceptual understanding with material execution created what might be recognized as complete creative agency, decision-making authority distributed throughout the entire making process rather than concentrated in separate planning phases.

\vspace{0.5cm}

The appearance of industrial production altered the foundations of these relationships of agency by introducing systematic specialization. Frederick Winslow Taylor's principles of scientific management exemplified this transformation, advocating for the concentration of knowledge in management roles while reducing workers to executors of predetermined procedures. As Taylor argued, efficiency required 'the managers assume, for instance, the burden of gathering together all of the traditional knowledge which in the past has been possessed by the workmen and then of classifying, tabulating, and reducing this knowledge to rules, laws, and formulae which are immensely helpful to the workmen in doing their daily work' (Taylor 1919). This extraction and centralization of craft knowledge created the foundation for what Harry Braverman later identified in \textit{Labor and Monopoly Capital} (1974) as the systematic separation of conception from execution, a division that fundamentally altered the relationship between thinking and making that had characterized traditional craft practice.

\vspace{0.5cm}

However, this transformation did not proceed unopposed. From the direct resistance of Luddism against mechanization to the reformist proposals of the Arts \& Crafts movement, various social movements emerged to challenge the dehumanization of productive work. The Luddites, British textile workers active between 1811-1816, responded to industrial mechanization not through blanket opposition to technology, but as historian Malcolm I. Thomis observed, because machine-breaking represented a strategic form of 'collective bargaining by riot' (Thomis 1993) when orthodox negotiation was impossible due to restrictive anti-union legislation and the scattered nature of industrial work. As Thomis documented, 'machine-breaking was, of course, by no means a new phenomenon when it appeared in Nottinghamshire in March 1811, being almost a time-honoured tradition among certain occupational groups' (Thomis 1993) used to 'effectively and quickly strike at an offensive local employer' (Thomis 1993). Crucially, Thomis noted that 'these attacks on machines did not imply any necessary hostility to machinery as such; machinery was just a conveniently exposed target against which an attack could be made' (Thomis 1993). As Lord Byron argued in Parliament during the height of Luddite activity, these 'outrages must be admitted to exist to an alarming extent, it cannot be denied that they have arisen from circumstances of the most unparalleled distress' (Byron 1812), driven by 'absolute want' when skilled craftsmen found 'their own means of subsistence were cut off' by mechanization (Byron 1812).

\vspace{0.5cm}

The Arts \& Crafts movement, emerging later in the century, offered a more sustained intellectual critique of industrial production's effects on creative agency. John Ruskin, the movement's intellectual foundation, argued in \textit{The Stones of Venice} that industrial mechanization represented a fundamental assault on human dignity, writing that ``we have much studied and much perfected, of late, the great civilized invention of the division of labour; only we give it a false name. It is not, truly speaking, the labour that is divided; but the men: Divided into mere segments of men, broken into small fragments and crumbs of life'' \citep{ruskin1892}. William Morris, inspired by Ruskin's critique, sought to restore what he called ``art which is made by the people and for the people, as a happiness to the maker and the user \citep{morris1996}'', advocating for production methods that would reunite intellectual conception with manual execution. Morris believed that creative work should demonstrate ``obvious traces of the hand of man guided directly by his brain, without more interposition of machines than is absolutely necessary to the nature of the work done'' \citep{morris1882}.

\vspace{0.5cm}

These movements shared a fundamental concern: the industrial division of labor threatened not merely economic arrangements but the essential human capacity for creative agency. Walter Crane, first president of the Arts and Crafts Exhibition Society, articulated this critique in \textit{The Claims of Decorative Art} (1892), arguing that ``the apotheosis of commercialism meant the degradation of art'' and lamenting that under industrial conditions ``there can be no possibility of the pleasure of the craftsman in fashioning his work, to give it the individual twist and play of fancy, the little touch of grace and ornamental feeling springing from the organic necessities of the work which is characteristic of the times when art and handicraft were united and living.'' Crane specifically attacked how industrialization had created a world where ``all the useful labours are made either terrible by long hours, or emptied of all joy and interest by being reduced to mechanism'' \citep{crane1892}. His vision opposed the industrial reduction of workers to mere components, advocating instead for the reunification of art and handicraft that had been systematically divided by mechanization.

\vspace{0.5cm}

Yet despite their moral urgency, these resistance movements could not halt the broader trajectory toward systematic separation of conception from execution that would later characterize digital fabrication workflows.

\section{Contemporary Digital Workflows: Extending Historical Fragmentation}

Contemporary digital design workflows extend this historical fragmentation into new technological domains, perpetuating the separation Braverman observed through software architectures and computational processes. The traditional craftsperson's embodied knowledge, held in hands, eyes, and intuitive understanding of materials, becomes progressively abstracted through layers of digital mediation. Modern CAD/CAM systems create distinct operational phases: human conceptualization, software translation, machine execution, and material output. Each transition representing a potential loss of agency, as the maker's intentionality becomes increasingly distant from the final artifact.

\vspace{0.5cm}

Contemporary digital design workflows perpetuate this historical fragmentation through their fundamental architecture and representational logic. Current CAD/CAM systems create distinct operational phases that mirror the industrial separation Braverman documented: human conceptualization in design software, algorithmic translation happens through file processing, machine execution follows predetermined paths, and material output emerges as the final step. Each transition represents a potential loss of agency, as the maker's intentionality becomes increasingly mediated through technological intermediaries that may not preserve the original creative reasoning behind design decisions.

\vspace{0.5cm}

Research \textit{Embodied Knowledge in Digital Spaces: Towards Human-Centered Fabrication Formats} suggests that ``current representation formats prioritize geometric precision over embodied knowledge, reducing complex creative processes to coordinates and mechanical instructions'' \citep{munoz2025}. The dominance of formats like G-code, which controls CNC machines and 3D printers through standardized commands, exemplifies how digital workflows eliminate the experiential knowledge that craftspeople traditionally embedded within their making processes. These technical standards capture precise geometric specifications but cannot encode the tacit understanding, material sensitivity, or adaptive decision-making that characterized unified craft practice.

\vspace{0.5cm}

This technological fragmentation operates at multiple levels simultaneously. Beyond the limitations of individual file formats, entire workflow architectures perpetuate the conception-execution divide through their structural organization. Contemporary digital fabrication requires users to navigate between specialized software environments: Computer Aided Design applications for conceptualization, Computer Aided Manufacturing programs for toolpath generation, and machine-specific control interfaces for execution. Each software transition introduces potential ``breakdown points'', moments where creative flow encounters systemic resistance or translation errors. The cumulative effect transforms the continuous dialogue between maker and material that characterized traditional craft into a series of discrete, mediated steps where creative agency becomes attenuated with each technological translation.

\section{Mapping Agency: From Unified to Fragmented Control}

The historical trajectory traced from medieval guilds through industrial mechanization to contemporary digital workflows reveals a systematic redistribution of creative control that demands more precise analytical examination. This transformation represents not merely technological evolution, but a restructuring of how decision-making authority flows through productive processes. To understand the implications for contemporary digital fabrication, it becomes necessary to map these different configurations of agency as distinct organizational forms, each embodying particular relationships between human intention, technological mediation, and material execution.

\subsection{Unified Agency: The Craftsperson's Integrated Practice}

Traditional craft practice operated through what this research terms unified agency, a configuration where conceptual understanding, material manipulation, and adaptive decision-making remain integrated within the craftsperson's direct control. This represents more than the romantic notion of ``hands-on'' making; it constitutes a particular organizational form where creative authority flows through continuous feedback loops between intention and execution.

\vspace{0.5cm}

David Pye's concept of the ``workmanship of risk'' captures this configuration precisely: ``the quality of the result is not predetermined, but depends on the judgment, dexterity and care which the maker exercises as he works'' \citep{pye1971}. The craftsperson's tools function as what Andy Clark and David Chalmers describe as extensions of an ``extended mind''; transparent intermediaries that amplify human capability without introducing systematic barriers between creative intention and material response \citep{clark1998}.

\vspace{0.5cm}

Crucially, unified agency enables what might be called adaptive authority: the capacity to modify design decisions based on material feedback, unexpected discoveries, or emergent possibilities that arise during the making process. The medieval carpenter adjusting joint techniques based on wood grain patterns exemplifies this adaptive authority, where creative control remains responsive to material conditions rather than predetermined by separate planning phases.

\vspace{0.5cm}

\textit{[Figure 1: Unified Agency diagram]}

\subsection{Distributed Agency: Industrial and Digital Fragmentation}

The industrial transformation traced through Taylorism and later digital workflows created what this research identifies as distributed agency: a configuration where creative control becomes systematically fragmented across separate operational domains, each governed by different logical systems and often different human operators. This represents a qualitatively different organizational form from unified agency, not merely a technological updating of traditional craft practice.

\vspace{0.5cm}

Distributed agency operates through what theorists recognize as sequential specialization: creative authority becomes divided among discrete phases where each stage operates according to its own technical logic and constraints \citep{enfield2017}. The Taylorist separation of planning from execution established this pattern, but digital workflows extend it through technological mediation that introduces additional layers of fragmentation.

\vspace{0.5cm}

Contemporary CAD/CAM systems exemplify this distributed configuration through their multi-stage architecture: conceptual design occurs within parametric modeling environments, geometric processing happens through file format translations, toolpath generation follows manufacturing optimization algorithms, and material execution proceeds through machine control protocols. Each stage embeds particular assumptions about what constitutes relevant information, creating systematic filtering that may eliminate precisely the kinds of adaptive responses that characterized unified agency.

\vspace{0.5cm}

\textit{[Figure 2: Distributed Agency diagram]}

\subsection{The Loss of Adaptive Authority}

The critical difference between unified and distributed agency lies not simply in the number of technological intermediaries involved, but in the systematic elimination of what this research terms adaptive authority, being the capacity for real-time modification of creative decisions based on emergent conditions. Traditional craft practice embedded this adaptive capacity throughout the making process, enabling continuous negotiation between intention and material response.

\vspace{0.5cm}

Distributed agency, by contrast, concentrates adaptive authority within the initial design phase while rendering subsequent stages increasingly deterministic. Once geometric specifications are locked into CAD files and translated through manufacturing software, the system resists the kinds of responsive modifications that characterized traditional craft dialogue between maker and material. This represents what Terry Winograd identified as ``breakdown points'' \citep{winograd1986}, moments where the technological system's logical structure conflicts with the adaptive nature of creative practice.

\vspace{0.5cm}

The emergence of increasingly sophisticated computational processes, from parametric optimization to AI-generated design variations, intensifies rather than resolves this fundamental structural limitation. These developments may increase the sophistication of initial design exploration, but they operate within the same distributed architecture that systematically concentrates creative authority in separate planning phases while rendering material execution increasingly automated and non-responsive.

\section{Toward Alternative Configurations}

Understanding agency as organizationally configured rather than technologically determined suggests possibilities for alternative approaches to digital fabrication. Rather than accepting the distributed model as inevitable, this research explores whether digital systems might be structured to preserve adaptive authority throughout the making process, enabling the continuous dialogue between intention and material response that characterized unified craft practice while leveraging the capabilities of contemporary fabrication technologies.

\vspace{0.5cm}

\textit{[Figure 3: Toward Integrated Digital Agency diagram]}

\section{Reclaiming Agency Within Technological Constraints}

This historical trajectory from unified craft practice through industrial fragmentation to contemporary digital workflows reveals that the challenge facing makers today is not technological limitation but organizational structure. The tools themselves: CNC machines, 3D printers, parametric design software represent unprecedented capabilities for material manipulation and geometric exploration. Yet their integration within distributed agency frameworks systematically eliminates the adaptive authority that enabled traditional craftspeople to maintain creative control throughout the making process.

\vspace{0.5cm}

The question is not whether to embrace or reject these technological capabilities, but how to reorganize their deployment in ways that restore human creative agency. Current approaches to fabrication ``democratization'' have focused primarily on access, making industrial machines smaller, cheaper, and more widely available, without addressing the fundamental workflow architectures that perpetuate the separation of conception from execution that Braverman identified as central to industrial production methods.

\vspace{0.5cm}

This represents a false choice between technological sophistication and human creativity. The medieval craftsperson's unified agency emerged not from the absence of tools but from organizational structures that preserved adaptive authority throughout the making process. Contemporary digital systems possess far greater material capabilities than traditional hand tools, yet their current organizational deployment fragments rather than enhances human creative control.

\vspace{0.5cm}

The path forward requires a tactical appropriation, working within existing technological ecosystems while reorganizing their operational logic to restore the continuous dialogue between intention and material response that characterized traditional craft practice. This means developing approaches that leverage computational precision and automated execution while preserving the maker's capacity for real-time adaptation, contextual response, and embodied decision-making.

\vspace{0.5cm}

Rather than rejecting digital fabrication systems, the following chapters of this research will explore how their representational frameworks might be restructured to honor what Walter Crane described as the craftsperson's capacity ``to give it the individual twist and play of fancy, the little touch of grace and ornamental feeling springing from the organic necessities of the work'', those responsive creative decisions that emerge when maker and material remain in continuous dialogue throughout the making process. Unlike the Luddites' strategic machine-breaking or the Arts \& Crafts movement's wholesale rejection of industrial methods, this approach recognizes that contemporary technological capabilities need not inevitably fragment creative agency. The challenge lies not in the machines themselves but in reorganizing how they are deployed within making workflows. The following chapters examine how such appropriation might operate in practice, suggesting pathways toward post-industrial craft, making practices that combine contemporary technological capabilities with organizational structures that preserve human creative agency without requiring retreat from digital fabrication's material possibilities.