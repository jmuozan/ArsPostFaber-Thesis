\clearpage

% Set chapter counter to 5 and reset section counter
\setcounter{chapter}{5}
\setcounter{section}{0}

% Add the chapter to table of contents
\addcontentsline{toc}{chapter}{\numberline{5}Conclusion. Questioning the Future of Human-Machine Interaction}

% Set up page style for this chapter
\pagestyle{fancy}
\fancyhf{} % Clear all header and footer fields
\fancyhead[L]{\footnotesize\textit{Ars Post Faber: Digital Fabrication Democratization Through Embodied Knowledge Preservation}}
\fancyfoot[C]{\thepage} % Page number in footer center
\renewcommand{\headrulewidth}{0pt}
\renewcommand{\footrulewidth}{0pt}

% Custom chapter title to match abstract formatting
\noindent
{\Large\textbf{Chapter 5: Conclusion. Questioning the Future of Human-Machine Interaction}}
\vspace{0.3cm}
\hrule
\vspace{0.8cm}
\label{ch:conclusions}

% Set no paragraph indentation
\setlength{\parindent}{0pt}

\textit{The future of making lies not in perfecting our machines, but in reimagining our relationship with them.}

\vspace{0.5cm}

This research emerged from the simple observation that despite unprecedented capabilities and accessibility, personal fabrication often feels disconnected from the intuitive creative processes that have characterized traditional making. The investigation done in this work following the assumption has highlighted that this disconnection reflects deeper assumptions about how humans should interact with computational systems, giving more importance to machine logic over human cognition, efficiency over creative agency, and predetermined procedures over adaptive response.

\section{Beyond Technological Solutions}

The historical trajectory traced through this work demonstrates that the challenges facing contemporary digital fabrication are not primarily technological but organizational. The industrial separation of conception from execution that Harry Braverman identified continues to manifest itself in digital workflows through CAD/CAM systems that privilege computational precision over embodied knowledge and adaptive decision-making.

\vspace{0.5cm}

Through this analysis it got revealed that the current distributed agency model widely adopted represents the aplicaion of this division. The craftsperson's unified agency, where creative authority flowed through continuous feedback loops between intention and execution, provided a contrasting organizational model that highlighted what has been systematically eliminated through successive technological "advances." This historical perspective suggested that future progress required questioning not just how to make better tools, but how the creative relationships with these tools is organized.

\section{The Limits of Access-Based Democratization}

Through the posterior examination of current fabrication "democratization" it got exposed the inadequacy of approaches that expand tool availability without addressing the workflow architectures that perpetuate creative fragmentation and agency assignation. While FabLabs and maker spaces have achieved great success in expanding access to sophisticated fabrication capabilities, they have largely reproduced the same organizational structures that separated conception from execution. To answer this, this work developed a preservation-based alternative that recognized genuine democratization requires the distribution of decision-making authority throughout creative processes instead of just expanded access to expert-designed systems.

\section{Experimental Insights: What Works and What Doesn't}

The experimental interventions provided specific insights about new pathways for preserving creative agency within digital fabrication. \textit{CR3ATED} demonstrated that alternative interface designs that allowed new ways of human-software-machine interaction could preserve material relationships while engaging with digital capabilities, but also showcased the expressive limitations that grow in simplification strategies.

\vspace{0.5cm}

Following up, the exploration of LLMs showed how language-mediated interaction could bridge between natural human expression and computational implementation, while highlighting critical distinctions between automation that diminishes agency versus augmentation that enhances creative capabilities. But perhaps most significantly, the \textit{AI.RTISANSHIP} intervention revealed the fundamental limitations of computational approaches to embodied knowledge preservation, exposing dimensions of embodied knowledge that systematically resist an algorithmic categorization.

\vspace{0.5cm}

These experiments collectively pointed towards the importance of organizational rather than technological innovation. Effective preservation of creative agency requires structures that maintain continuous dialogue between maker, material, and machine rather than just systems that try to capture and replay predetermined procedures.

\section{Unclosed Workflows as Methodological Innovation}

The development of \textit{Ars Post Faber} synthesized the experimental insights in a concrete exploration of how digital design and fabrication tools might preserve unified agency, with the plugin looking to demonstrate that the conception-execution separation is not technologically inevitable.

\vspace{0.5cm}

The unclosed workflow approach represented a challenge to certain assumptions about how digital fabrication must operate. By maintaining live connections between digital design, physical fabrication, and human decision-making, these workflows attempt tp preserve the maker's hability for real-time adaptation, contextual modification, and responsive intervention throughout the entire making process.

\vspace{0.5cm}

Through the integration of more human ways of interaction (in the form of components in the CAD software) a fluid making workflow was looked to be created, where the boundary between physical and digital work was deleted, attempting to argue that the democratization of making requires preserving continuity of creative decision-making across all available tools and contexts.

\section{Questions for Future Investigation}

Following the analyssis and developement done, this research concludes not with definitive answers but with questions that can only be anwered through sustained community engagement. Several questions emerged that require ongoing investigation: Can interaction paradigms that preserve creative agency scale beyond individual maker contexts? Do alternative human-machine interaction approaches genuinely support the development of embodied knowledge, or do they simply create new forms of technological dependency? How might emerging technologies be deployed to support creative agency within making processes? Through the provided vocabulary for discussing these challenges developed in the theoretical framework and the promising pathways following the experimental investigations this work expects to create a starting point to debate and experiment, hopefully trying to find answers to these questions.

\vspace{0.5cm}

At the end, the resolution of these questions depends heavily on how they are taken up within FabLabs, makerspaces, and educational institutions, where new approaches to human-machine interaction can be tested against actual creative needs and social contexts. These spaces possess the infrastructure, community connections, and experimental approaches necessary for evaluating whether the alternative paradigm proposed in this work genuinely enhances creative practice or just adds complexity to existing workflows.

\section{An Invitation to Continued Exploration}

This research positions itself as one experimental approach among many possible directions for reimagining human-machine interaction within creative contexts. The theoretical insights, experimental findings, and technical implementations should be understood as propositions for testing rather than definitive solutions to identified challenges.

\vspace{0.5cm}

The future of human-machine interaction within creative contexts will emerge through collective experimentation rather than individual innovation. Only through sustained engagement by diverse communities of makers, educators, technologists, and scholars will it be possible to discover how these questions might be productively addressed.

\vspace{0.5cm}

This experimental spirit represents the understanding that the future of human-machine interaction remains open for creative reimagining by those willing to question current assumptions and explore alternative possibilities. The conversation continues in workshops, makerspaces, and fab labs around the world, where makers will ultimately determine whether these ideas contribute to more human-centered approaches to technology.

\vspace{0.5cm}

The research most important contribution may be the questions it raises rather than the answers it provides. How might we completely reimagine the relationship between humans and computational systems? What would digital fabrication look like if designed from the to preserve instead of eliminating creative expression? How can we ensure that technological advancement enhances rather than diminishes human creative potential? Questions that extend beyond fabrication to broader concerns about how emerging technologies affect human agency, creativity, and autonomy within an increasingly computational world. Only time and community engagement will provide the definitive evaluation of whether this research direction proves productive in our evolving technological landscape.